\documentclass[a4paper,12pt, twoside]{article}
%\documentclass[a4paper,12pt, twoside]{book}

\usepackage[papersize={210mm,297mm},tmargin=20mm,bmargin=20mm,lmargin=20mm,rmargin=20mm]{geometry}

\usepackage[utf8]{inputenc}
%https://mirror.hmc.edu/ctan/macros/latex/contrib/babel-contrib/turkish/turkish.pdf
\usepackage[english]{babel}
%\usepackage[T1]{fontenc}

\usepackage{amsmath,amssymb,mathabx}%\for eqref
\usepackage{lscape}

\usepackage{hyperref}
\hypersetup{
    colorlinks,
    citecolor=black,
    filecolor=black,
    linkcolor=blue,
    urlcolor=red}
  

%%% \usepackage{svg}
%%% https://tex.stackexchange.com/questions/122871/include-svg-images-with-the-svg-package/129854
\usepackage{graphicx}
\graphicspath{ {./figurler/} }

\usepackage[colorinlistoftodos]{todonotes}
\usepackage{fancyhdr}

\usepackage{indentfirst}
%% paragraf girintisi
\setlength{\parindent}{5ex}

%% Daha sonra yazılacak kısımları not düşmek için...
\newcommand{\YAZILACAK}{{\vspace{18pt}\bf\Large \color{red} YAZILACAK}}


\pagestyle{fancy}
\fancyhf{}
\lhead{ Kuantum Fiziği }
\chead{\thepage}
\rhead{Mesut Karakoç}
\lfoot{Akdeniz Üniversitesi}
\cfoot{}
%\rfoot{BF}

\title{Akdeniz Üniversitesi\\ Fen Fakültesi - Fizik Bölümü\\FİZ319 Kuantum Fiziği Ders Notları}

\author{\setlength{\unitlength}{6mm}
\begin{picture}(10,10)
\put(1.1,0){\includegraphics[width=4.5cm]{Leptonic_event_in_Gargamelle_bubble_chamber.jpg}}
\end{picture} \\ Doç. Dr. Mesut Karakoç}


\date{\today}

\begin{document}

%% Turkish babel problem
%% https://tex.stackexchange.com/questions/160385/newgeometry-doesnt-work-with-turkish-babel-package
%%\shorthandoff{=}% Make = not active any more

\maketitle

\newpage

% change name to "İçindekiler"
\renewcommand{\contentsname}{İçindekiler}
\tableofcontents{}

\listoffigures
 
\listoftables

\newpage

{
\hspace{.5\textwidth}
\begin{minipage}{.5\textwidth}
\raggedleft
If all this damned quantum jumps were really to stay, I should be
sorry I ever got involved with quantum theory.

—Erwin Schrödinger
\cite{book:Ficek}

%% Latince için
%% post iacturam quis non sapit!
%% Who is not wise after he has lost something?
%% https://quizlet.com/23756827/latin-proverbs-h-flash-cards/
\end{minipage}
}

\setcounter{section}{3} %% THIS WILL BE DELETED when all chapters merged!
\section{Bir Boyutlu Potansiyeller}

Üç boyutlu bir evrende yaşıyor olmamıza rağmen, bir çok fiziksel olayı (hareketi) bir boyutlu olarak tanımlamak mümkündür. Bu nedenle bu bölümde klasik fiziğin açıklayamadığı fakat kuantum fiziğiyle çalışabildiğimiz bazı bir boyutlu sistemleri inceleyeceğiz.


\subsection{Basamak Potansiyeli}
%%
\begin{figure}[hbtp]
	\centering
	\includegraphics[width=0.7\linewidth]{figurler/Basamak_Potansiyeli}
	\caption{Basamak potansiyeli.}
	\label{fig:basamakpotansiyeli}
\end{figure}
%%
Basamak potansiyeli için bir örnek yukarıdaki şekildeki gibi olur. Şekilden anlaşılacağı üzere basamak potansiyeli; birbirinden farklı sabit potansiyellere sahip iki bölge içeren bir durumdur. Bir boyutlu hali için matematiksel ifadesi aşağıdaki gibidir.
%%
\begin{equation}
V ( x )  = \left\{ 
\begin{array} { l l } 
{ 0 } & {\Leftarrow x < 0 } \\ 
{ V _ { 0 } } & {\Leftarrow x \geq 0 } 
\end{array} \right. 
\end{equation}
%%

Bu potansiyeli zamandan bağımsız Schrödinger denklemi ile çalışabiliriz. Öncelikle Schrödinger denklemini yazılışı daha kolay olan,
%%
\begin{equation}
- \frac { \hbar ^ { 2 } } { 2 m } \frac { d ^ { 2 } u ( x ) } { d x ^ { 2 } } + V ( x ) u ( x ) = E u ( x )
\end{equation}
%%
formuna dönüştürebiliriz.

\begin{equation}
\frac { d ^ { 2 } u ( x ) } { d x ^ { 2 } } + \frac { 2 m } { \hbar ^ { 2 } } [ E - V ( x ) ] u ( x ) = 0
\end{equation}
%%
Basamak potansiyelinin değerinin sıfır olduğu bölge için,
%%
\begin{equation}
\frac { 2 m E } { \hbar ^ { 2 } } = k ^ { 2 }
\end{equation}
%%
tanımını ve sıfırdan farklı olduğu bölge için
%%
\begin{equation}
\frac { 2 m \left( E - V _ { 0 } \right) } { \hbar ^ { 2 } } = q ^ { 2 }
\end{equation}
%%
tanımını yapabiliriz.

\begin{equation}
u ( x ) = e ^ { i k x } + R e ^ { - i k x }
\end{equation}


\begin{equation}
\begin{array} { r l } 
{ j } & { = \frac { \hbar } { 2 i m } \left( u ^ { * } \frac { d u } { d x } - \frac { d u ^ { * } } { d x } u \right) = \frac { \hbar } { 2 i m } \left[ \left( e ^ { - i k x } + R ^ { * } e ^ { i k x } \right) \left( i k e ^ { i k x } - i k R e ^ { - i k } \right) - c . c \right] } \\ 
{ } & { = \frac { \hbar k } { m } \left( 1 - | R | ^ { 2 } \right) } \end{array}
\end{equation}

\begin{equation}
u ( x ) = T e ^ { i q x }
\end{equation}





\begin{equation}
j = \frac { \hbar q } { m } | T | ^ { 2 }
\end{equation}



\begin{equation}
\frac { \hbar k } { m } \left( 1 - | R | ^ { 2 } \right) = \frac { \hbar q } { m } | T | ^ { 2 }
\end{equation}



\begin{equation}
1 + R = T
\end{equation}



\begin{equation}
\begin{array} { r l } { \left( \frac { d u } { d x } \right) _ { s } - \left( \frac { d u } { d x } \right) _ { - \varepsilon } } & { = \int _ { - \varepsilon } ^ { \varepsilon } d x \frac { d } { d x } \frac { d u } { d x } } \\ { } & { = \int _ { - s } ^ { s } d x \frac { 2 m } { \hbar ^ { 2 } } [ V ( x ) - E ] u ( x ) = 0 } \end{array}
\end{equation}




\begin{equation}
\begin{aligned} \left( \frac { d u } { d x } \right) _ { a + s } - \left( \frac { d u } { d x } \right) _ { a - s } & = \frac { 2 m } { \hbar ^ { 2 } } \int _ { a - s } ^ { a + s } d x \lambda \delta ( x - a ) u ( x ) \\ & = \frac { 2 m } { \hbar ^ { 2 } } \lambda u ( a ) \end{aligned}
\end{equation}


\begin{equation}
i k ( 1 - R ) = i q T
\end{equation}


\begin{equation}
\begin{array} { l } { R = \frac { k - q } { k + q } } \\ { T = \frac { 2 k } { k + q } } \end{array}
\end{equation}


\begin{equation}
\begin{array} { l } { \frac { \hbar k } { m } | R | ^ { 2 } = \frac { \hbar k } { m } \left( \frac { k - q } { k + q } \right) ^ { 2 } } \\ { \frac { \hbar q } { m } | T | ^ { 2 } = \frac { \hbar k } { m } \frac { 4 k q } { ( k + q ) ^ { 2 } } } \end{array}
\end{equation}


\begin{equation}
u ( x ) = T e ^ { - | q | x }
\end{equation}


\begin{equation}
| R | ^ { 2 } = \left( \frac { k - i | q | } { k + i | q | } \right) \left( \frac { k - i | q | } { k + i | q | } \right) ^ { * } = 1
\end{equation}


\begin{equation}
T = \frac { 2 k } { k + i | q | }
\end{equation}


\subsection{Sonlu Potansiyel Kuyusu}

\newpage
% In the preamble, add "\renewcommand\refname{New Title}" for article type documents 
% and "\renewcommand\bibname{New Title}" for book and report type documents.
\renewcommand\refname{Kaynaklar}
\bibliography{quantumBIB}{}
%% https://www.sharelatex.com/learn/latex/bibtex_bibliography_styles
 \bibliographystyle{plain}
%% \bibliographystyle{alpha}
%%\bibliographystyle{apalike}
\end{document}

