\documentclass[a4paper,12pt, twoside]{article}
%\documentclass[a4paper,12pt, twoside]{book}

\usepackage[papersize={210mm,297mm},tmargin=20mm,bmargin=20mm,lmargin=20mm,rmargin=20mm]{geometry}

\usepackage[utf8]{inputenc}
%https://mirror.hmc.edu/ctan/macros/latex/contrib/babel-contrib/turkish/turkish.pdf
\usepackage[english]{babel}
%\usepackage[T1]{fontenc}

\usepackage{amsmath,amssymb,mathabx}%\for eqref
\usepackage{lscape}

\usepackage{hyperref}
\hypersetup{
    colorlinks,
    citecolor=black,
    filecolor=black,
    linkcolor=blue,
    urlcolor=red}
  

%%% \usepackage{svg}
%%% https://tex.stackexchange.com/questions/122871/include-svg-images-with-the-svg-package/129854
\usepackage{graphicx}
\graphicspath{ {./figurler/} }

\usepackage[colorinlistoftodos]{todonotes}
\usepackage{fancyhdr}

\usepackage{indentfirst}
%% paragraf girintisi
\setlength{\parindent}{5ex}

\pagestyle{fancy}
\fancyhf{}
\lhead{ Kuantum Fiziği }
\chead{\thepage}
\rhead{Mesut Karakoç}
\lfoot{Akdeniz Üniversitesi}
\cfoot{}
%\rfoot{BF}

\title{Akdeniz Üniversitesi\\ Fen Fakültesi - Fizik Bölümü\\FİZ319 Kuantum Fiziği Ders Notları}

\author{\setlength{\unitlength}{6mm}
\begin{picture}(10,10)
\put(1.1,0){\includegraphics[width=4.5cm]{Leptonic_event_in_Gargamelle_bubble_chamber.jpg}}
\end{picture} \\ Doç. Dr. Mesut Karakoç}


\date{\today}

\begin{document}

%% Turkish babel problem
%% https://tex.stackexchange.com/questions/160385/newgeometry-doesnt-work-with-turkish-babel-package
%%\shorthandoff{=}% Make = not active any more

\maketitle

\newpage

% change name to "İçindekiler"
\renewcommand{\contentsname}{İçindekiler}
\tableofcontents{}

\listoffigures
 
\listoftables

\newpage

{
\hspace{.5\textwidth}
\begin{minipage}{.5\textwidth}
\raggedleft
If all this damned quantum jumps were really to stay, I should be
sorry I ever got involved with quantum theory.

—Erwin Schrödinger
\cite{book:Ficek}

%% Latince için
%% post iacturam quis non sapit!
%% Who is not wise after he has lost something?
%% https://quizlet.com/23756827/latin-proverbs-h-flash-cards/
\end{minipage}
}

\setcounter{section}{1} %% THIS WILL BE DELETED when all chapters merged!
\section{Dalga-Parçacık İkilemi ve Schrödinger Denklemi}

Kuantum fiziğinin doğum sürecini anlattığımız kısım genellikle \emph{Eski Kuantum Teorisi} olarak adlandırılır. Gerçekleştirilen keşifleri açıklamak için  kullanılan veya ortaya konan kuralların tam olarak birbirleriyle sağlam bir bağlantısının olduğunu söylemek pek mümkün değildir. Daha iyi bir açıklama için ortaya konan ``Kuantum Mekaniği" iki defa keşfedilmiştir denebilir, ilki 1925'te matris mekaniği formalizmiyle Werner Heisenberg tarafından ve ikincisi 1926'da dalga mekaniği ile Erwin Schrödinger tarafındandır. Her ikisinin eş değer olduğu sonradan gösterilmiş olmasına rağmen Schröndinger'in yöntemi daha çok kullanılır hale gelmiştir. Çünkü dalga mekaniğinin matematiği fizikçiler arasında daha yaygındı \cite{book:Gasiorowicz}.


\subsection{Dalga-Parçacık İkilemi}


\subsection{Yeni Başlık}








\newpage
% In the preamble, add "\renewcommand\refname{New Title}" for article type documents 
% and "\renewcommand\bibname{New Title}" for book and report type documents.
\renewcommand\refname{Kaynaklar}
\bibliography{quantumBIB}{}
%% https://www.sharelatex.com/learn/latex/bibtex_bibliography_styles
 \bibliographystyle{plain}
%% \bibliographystyle{alpha}
%%\bibliographystyle{apalike}
\end{document}

